




\section{Survey of JVM-based Actor implementations}
\subsection{SALSA}
SALSA (Simple Actor Language System and Architecture) is an actor-based programming language which is basically designed for mobile and Internet computing \cite{varela2001pdr}. SALSA (version 1.1.2) is well suited for dynamically reconfigurable open and distributed applications. In addition to basic features of an Actor-based language such as active objects and asynchronous message passing, SALSA also provides universal naming, migration, location transparency and advanced concurrency constructs for coordination \cite{SALSA_Man}.

SALSA is a dialect of Java and tries to use features and APIs provided by Java\cite{SALSA_Man}. Java objects and primitive types are all accessible in SALSA actors. Every actor is specified as a \textit{behavior}. SALSA compiler transforms these behaviors into Java code and then the Java compiler generates the byte-code.
\begin{verbatim}
behavior Threadring {
    void passToken() {
        // .... 
    }
}
\end{verbatim}
For actor creation SALSA uses the \code{new} keyword which is also used for creating normal object. In addition for sending a message to an actor SALSA provides the following syntax:
\begin{verbatim}
    actorRef <- messageName(messageParams...);
\end{verbatim}
In SALSA, actor addresses are passed by reference, while objects are transferred by making a copy.

Furthermore, SALSA has its own syntax for coordinating concurrency. It has three constructs for this purpose: token-passing continuation, join blocks and first-class continuation.\\
Token-passing continuations are designed to specify a partial order of message processing \cite{SALSA_Man}.
\begin{verbatim}
    standardOutput <− print( "Hello " ) @
    standardOutput <− print( "World" ) ;
\end{verbatim}
Token-passing continuation ensures that the order of message processing. For instance, in the above code the first message would be processed and then token is passed to the next message to continue processing. In addition, programmer can explicitly use the token as a result to pass on to a function for further processing. 
Moreover, join blocks can be used to specify a barrier for parallel processing activities and join their results in a subsequent message. In the case of join blocks, every message inside the block gets processed and then continuation would pass to the statement after the join block. The following code snippet shows an example of using join blocks.
\begin{verbatim}
token f1;
token f2;
join {
    f1 = fib1 <- compute_fib(n-1);
    f2 = fib2 <- compute_fib(n-2);
} @ client <- result(f1+f2);
\end{verbatim}
First-class continuations delegate computation to a third party, enabling dynamic replacement or expansion of messages grouped by token-passing continuations. First-class continuations are very useful for writing recursive code \cite{SALSA_Man}.\\
SALSA provides four message properties that can be used with message sending: \code{priority}, \code{delay}, \code{waitfor}, and \code{delayWaitfor}. The syntax used to assign to a message a given property is the following, where \code{<property name>} can be either priority, delay, waitfor, and delayWaitfor: 
\begin{verbatim}
    actor<-myMessage(parameters...):<property name>
\end{verbatim}
During runtime, SALSA creates a Java thread for each actor created in the system. This makes a SALSA application heavyweight and expensive. Thread context switching is inefficient and consumes a lot of system resources. In addition, scheduling of these threads depends on the JVM, which could be different from one platform to the other.\\
Moreover, SALSA supports World Wide Computing (WWC) by providing a universal naming mechanism and a universal protocol for locating actors. The protocol used in SALSA is called Universal Actor Naming Protocol (UANP). 

\subsection{Actor Architecture}
Actor Architecture (AA) \cite{AA_Man} is another Java library for Actor programming. It is similar to ActorFoundry in some aspects. In AA (version 0.1.3) actors are executed on \textit{AA Platform}. AA Platform provides services for actors for message delivery, migration and middle actor services such as matchmaking and brokering. However, AA is not as configurable as ActorFoundry. In addition, AA does not have any mechanism for scheduling actors and it is dependent on JVM for scheduling actors which are wrapped into threads. Hence, it is like SALSA in this sense and it has the same inefficiencies. The syntax and semantic for asynchronous and synchronous communication are similar to ActorFoundry. However, AA does not support by reference messaging for locally delivered messages. In some cases by reference messaging does not break the semantics and it is very efficient. 


\subsection{Discussion}

Discuss Erlang, ThaL and Terracotta here.


\small{
% use packages: array
\begin{table}[h!b!p!]
\caption{Semantics Comparison}
\centerline{
\begin{tabular}{| p{2cm} | p{1.5cm} | p{1.3cm} | p{2cm} | p{1.3cm} | p{1.5cm} | p{2cm} |}
\hline
 & SALSA & Scala actors & Kilim & Actor Architecture & JavAct & ActorFoundry \\ 
\hline
Safe message-passing & Yes & No & No & Yes & No & Yes \\ 
\hline
Location transparency & Yes & No & No & Yes & Yes & Yes \\ 
\hline
Fairness & Yes & No & No & Yes & No & Yes \\ 
\hline
Mobility & Yes & No & No & Yes & Yes & Yes \\ 
\hline
Memory management & Automatic GC & × & × & Explicit & Explicit & Explicit \\ 
\hline
Actor mapping & JVM Threads & Light-weight Tasks & Continuations & JVM Threads & JVM Threads & Continuations \\
\hline
\end{tabular}
}
\label{semantics_comparison}
\end{table}
}
